\documentclass[10pt, english]{article}
\usepackage[T1]{fontenc}
\usepackage{babel}
\usepackage{amsfonts} 
\usepackage{amsmath}
\usepackage{graphicx}
\usepackage{hyperref}
\usepackage{titlesec}
\usepackage{changepage}
\usepackage{setspace}
\usepackage[margin=0.875in]{geometry}
\usepackage{multicol}
\usepackage{xcolor}

\begin{document}
\title{Reflections on Chapter 7}
\date{}
\author{}

\maketitle


\section*{General Thoughts}
This chapter extends the previous chapter in a pretty interesting direction. Keeping track of back doors and colliders etc seems pretty tedious to do by hand, and prone to errors, so it's nice that there's libraries out there for it. I'll have to keep that
in mind if I work for DAGs in the future. Personally the counterfactual model feels much more intuitive to me as a way to approach the problem, it really illustrates the idea of intevening in a clear way. But both approaches are very similar at the end of the day.


\section*{Space for future reading}
There's nothing in this chapter that nessecarily makes we want to dig into any research. It could be a good excercise to remember, and apply the DAG approach in a project in the future.

\end{document}