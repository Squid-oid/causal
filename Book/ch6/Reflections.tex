\documentclass[10pt, english]{article}
\usepackage[T1]{fontenc}
\usepackage{babel}
\usepackage{amsfonts} 
\usepackage{amsmath}
\usepackage{graphicx}
\usepackage{hyperref}
\usepackage{titlesec}
\usepackage{changepage}
\usepackage{setspace}
\usepackage[margin=0.875in]{geometry}
\usepackage{multicol}
\usepackage{xcolor}

\begin{document}
\title{Reflections on Chapter 6}
\date{}
\author{}

\maketitle


\section*{General Thoughts}
Structural equation models appear to be a pretty useful framework for assigning and communicating about models, as well as a nice way to think about your models, when 
hunting for backdoor paths, and how you need to condition. It seems scarily likely to create structures that become unsolvable/observable, both in theory, or caused by the nature of your observed data.


\section*{Space for future reading}
The frequent references to econometrics throughout this book makes me think I should check more on the products of that field. I should also look into TSEM construction, and what heuristics you can use to ensure that you
get a nice SEM that can be phrased in a lower triangular form.

\end{document}