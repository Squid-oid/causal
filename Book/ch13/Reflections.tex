\documentclass[10pt, english]{article}
\usepackage[T1]{fontenc}
\usepackage{babel}
\usepackage{amsfonts} 
\usepackage{amsmath}
\usepackage{graphicx}
\usepackage{hyperref}
\usepackage{titlesec}
\usepackage{changepage}
\usepackage{setspace}
\usepackage[margin=0.875in]{geometry}
\usepackage{multicol}
\usepackage{xcolor}

\begin{document}
\title{Reflections on Chapter 13}
\date{}
\author{}

\maketitle


\section*{General Thoughts}
This chapter covered some relatively specific models, which seems to be a bit of a theme
for the second portion of the book, but the next chapters seem a bit more interested to me.
The focus on IVs, proxies, and weak instruments is clearly very important for applied 
econometrics, but a bit removed from the rather kind problems that I've worked with so far.
This is again a chapter that motivates me to keep this book on my bookshelf, since if I need to work 
with IV models in the future, being able to flip this open to check up on how I can work with data,
and knowing that these tools exist is super useful, but I don't think I'll ever end up with these
moments in my head, more just the broad strokes and knowing that tools exist to tackle these problems, as
well as knowing that we have to be careful and precise in our handling of problems.

\section*{Space for future reading}
If I were to revisit this chapter, it would probably be to dig deeper into the weak-IV solutions,
since that captures both a lot of the core issues, as well as being surprisingly general, while still
being robust to essentially the worst case of data.

\end{document}