\documentclass[10pt, english]{article}
\usepackage[T1]{fontenc}
\usepackage{babel}
\usepackage{amsfonts} 
\usepackage{amsmath}
\usepackage{graphicx}
\usepackage{hyperref}
\usepackage{titlesec}
\usepackage{changepage}
\usepackage{setspace}
\usepackage[margin=0.875in]{geometry}
\usepackage{multicol}
\usepackage{xcolor}

\begin{document}
\title{Reflections on Chapter 4}
\date{}
\author{}

\maketitle


\section*{General Thoughts}
This chapter was a lot heavier in content, and with much more new content as opposed to previous chapters, a real step up. Neyman orthogonality seems really key to this whole book,
and also seems core to the authors research, so it's worth checking out in detail. The explanation is a bit sparse, and the proof of orthogonality on page 106, requires a bit of work 
to follow imo. I realize the book is written for graduates, or undergrads with a good deal of interest in the subject, but I really think a more thorough derivation and 
explanation of why you are allowed to treat the derivatives as you do, why the expectations become expectations of residuals, and how to think when approaching proving Neyman 
orthogonality would be super useful.

\section*{Space for future reading}
Absoluteltely the papers introducing Neyman orthogonality, this chapter also made me feel like I should go back and read a bit more about the foundational stuff in theoretical stats. 

\end{document}