\documentclass[10pt, english]{article}
\usepackage[T1]{fontenc}
\usepackage{babel}
\usepackage{amsfonts} 
\usepackage{amsmath}
\usepackage{graphicx}
\usepackage{hyperref}
\usepackage{titlesec}
\usepackage{changepage}
\usepackage{setspace}
\usepackage[margin=0.875in]{geometry}
\usepackage{multicol}
\usepackage{xcolor}

\begin{document}
\title{Reflections on Chapter 14}
\date{}
\author{}

\maketitle


\section*{General Thoughts}
Heterogeneous Treatment effects appear to be quite useful in real life, since we can often expect treatment effects to be based on confounders, and allocating treatment according to that heterogenity is desireable. The estimation techniques
for it in this chapter seems somewhat weak and fraught with pitfalls, so if you want this kind of granular estimation I suspect you need one or more of the following: very large and dense datasets, very high impact of heterogenity, or strong 
priors on which specific confounders are interesting, and in which ways. Luckily this might be the case more often than expected when we want this, in medicince trials for instance there are strong priors, and datasets are growing bigger
and better every day, as well as our ability to efficiently work with them. These tools do still seem a bit out of reach for the hobbyist in most cases, but they are again always good have encountered, and may very well come up in my professional
life, at which point having some understanding from this, and a quality reference to go to will be very useful.

\section*{Space for future reading}
Categorical heterogenity or low dim heterogenity are acheivable to work with in some smaller project alt. to read more about, but still capture a good deal of the important intuition and knowledge for these tools.

\end{document}